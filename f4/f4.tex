\documentclass{article}

\usepackage[T1]{fontenc}
\usepackage[utf8]{inputenc}
\usepackage[english]{babel}
\usepackage{amssymb}
\usepackage{xcolor}
\usepackage{url}

\newcommand{\grobner}{Gr\"obner}
\newcommand{\faugere}{Faug\'ere}
\newcommand{\code}[1]{\texttt{#1}}

\begin{document}
\title{F4: Computing Gröbner Bases Using F4}

\author{
  Alexander Demin\\
  MPI Informatics, Saarland Informatics Campus, Germany\\
  \texttt{alexander.demin.eternal@gmail.com}\\
  Hamid Rahkooy\\
  MPI Informatics, Saarland Informatics Campus, Germany\\
  \texttt{hrahkooy@mpi-inf.mpg.de}
  \and
  Thomas Sturm\\
  CNRS, Inria, and the University of Lorraine, France\\
  \texttt{thomas.sturm@cnrs.fr}
}

\date{January 2023}

\maketitle

\section{Introduction}

F4 is a package for the computation of \grobner{} Bases over rational numbers
using \faugere{}'s F4 algorithm \cite{F4}. It is compatible with existing REDUCE term
orderings, as used with the GROEBNER package and the CGB package. Some package usage examples are
presented in Section~\ref{SE:usage} and in the package test file
\code{f4.tst}.

The F4 algorithm is highly popular among practitioners and settled in the core of \grobner{} bases
computation routines in leading computer algebra systems. The efficiency of F4 partly comes from 
using Gaussian elimination for polynomial reductions, 
which allows the use of highly optimized linear algebra kernels in the implementation.

This package is migrated from the Julia package Groebner.jl version 0.2.10.
Groebner.jl is available at \url{https://github.com/sumiya11/Groebner.jl}, last accessed on February 2023.
The implementation utilizes several modern symbolic computation techniques, 
such as monomial hashtables, monomial divisibility masks, and specialized row-reduction algorithms.

\section{Package Usage}\label{SE:usage}

The following command loads the package into REDUCE:
%
\begin{center}
    \code{load\_package f4;}
\end{center}

The package provides the command \code{f4} and the command \code{torder} to manipulate the current
term order. It comes with some switches, which are described in Section~\ref{SE:switches}.

The principal function in the interface is
%
\begin{center}
    \code{f4(\{expr\_1, \ldots, expr\_m\}, \{var\_n, \ldots, var\_n\}, ord)}
\end{center}
%
It computes a \grobner{} basis of the polynomials \code {expr\_1}, \dots,~\code{expr\_m} in the variables 
\code{var\_1}, \dots,~\code{var\_n} using a
term order constructed from the given variables and the sort mode \code{ord}. As usual, all REDUCE
kernels can play the role of variables of the polynomial ring under consideration. F4 supports all sort modes
available through the torder command, including the basic orders \code{lex}, 
\code{gradlex}, and \code{revgradlex}, as well as block, weighted, graded, and matrix orders. 
Recall that the list of variables
specifies their relative order from largest to smallest, like $\code{var\_1} \succ \dots \succ
\code{var\_n}$.

The arguments \code{var\_1}, \dots,~\code{var\_n} and \code{ord} are optional. The default term
order is $\code{lex}$ with respect to all kernels occurring in numerators of \code{expr\_1},
\dots,~\code{expr\_m}, sorted by the REDUCE kernel order. Note that in this default case, variables
are assigned to the polynomial ring without being explicitly listed in \code{var\_1},
\dots,~\code{var\_n}.

Variables and sort mode can be globally specified using the command
%
\begin{center}
    \code{torder(vars, ord)}.
\end{center}
%
Using the optional arguments of \code{f4} shadows the current global
setting. Note that graded term orders like
\code{gradlex} and \code{revgradlex} tend to be the fastest for \grobner{}
basis computations \cite[\S5]{tolstaya}. By default, F4 inter-reduces the resulting basis, 
which makes it unique wrt.~the chosen term order.
%
\paragraph{Example}
On input of
\begin{verbatim}
1: load_package f4;

2: torder({x, y, z}, lex)$

3: f4({x*y + 1, y*z + 1});
\end{verbatim}
we obtain the \grobner{} basis
\code{\{y*z + 1, x - z\}} using lexicographic order with $\code{x} \succ
\code{y} \succ \code{z}$. A subsequent call
\begin{verbatim}
4: f4({x*y + 1, y*z + 1}, {z, y, x}, lex);
\end{verbatim}
with optional arguments shadows our \code{torder} setting above in favor of $
\code{z}
\succ \code{y} \succ \code{x}$. We then obtain the \grobner{} basis
\code{\{y*x + 1,z - x\}}.
%
\section{Switches}
\label{SE:switches}

\begin{description}
\item[\code{f4modular:}] Default is \code{on}. If set \code{on}, use modular algorithms
during computation. This only improves performance and does not affect the final result.
\end{description}

\section{Symbolic Mode Interface}

In symbolic mode the term order must be specified explicitly before \grobner{} basis computation.
%
\begin{description}\sloppy
\item[\code{torder(\{\{'list, var\_1, \ldots, var\_n\}, ord\}):}] with variables \code{var\_1},
\dots, \code{var\_n} and sort mode \code{ord} as described in Section~\ref{SE:usage}. The return
value is the old setting.
\end{description}

There are three alternative symbolic mode entry points for \grobner{} basis computation, depending
on the syntax of input and output polynomials:
\begin{description}
\item[\code{f4\_groebnerp(\{expr\_1, \ldots, expr\_m\}):}] \code{expr\_1}, \dots, \code{expr\_m} are
Lisp Prefix Forms. The return value is a list of polynomials as Lisp Prefix Forms, which are ordered
with respect to the chosen term order.
%
\item[\code{f4\_groebnerf(\{expr\_1, \ldots, expr\_m\}):}] \code{expr\_1}, \dots, \code{expr\_m} are
polynomials as Standard Forms. The return value is a list of polynomials as Standard Forms, which
are ordered with respect to the current REDUCE kernel order. Note that the information on the term
order gets lost here.
%
\item[\code{f4\_groebnerPoly(\{expr\_1, \ldots, expr\_m\}):}] \code{expr\_1}, \dots, \code{expr\_m}
are polynomials as Polynomial data structures of the F4 package. The return value is a list of
polynomials as such Polynomial data structures.
%
\item[\code{f4\_groebnerq(\{expr\_1, \ldots, expr\_m\}):}] \code{expr\_1}, \dots, \code{expr\_m} are
polynomials as Standard Quotients. The return value is a list
of polynomials as Standard Quotients, which are ordered with respect to the current REDUCE kernel
order. Note that the information on the term order gets lost here.
\end{description}
%
\section{Concluding remarks}

During the package implementation and benchmarking stage we have observed several features of REDUCE 
(in particular, in comparison to Julia) that we would like to outline here:
%
\begin{description}
	\item REDUCE Standard Quotient implementation of rational numbers is competitive to GNU GMP \cite{GMP}
    internally used by Julia on the most of our benchmarks.
	Still, expression swell occurring in F4 matrices is too severe to forgo modular computations;
	%
	\item The best strategy for Gr{\"o}ebner bases modular computation usually is 
	to pick the largest possible ``register size'' moduli, since fast modular arithmetic could be implemented then. 
    The portability guarantees of REDUCE impose a limit on the maximal safe size of an unboxed integer, 
    which might complicate selection of large enough moduli.
	In our implementation, we select the size of the moduli dynamically based on 
	the platform we are running on.
	% 
	\item The fact that REDUCE package needs to support a multitude of 
	possible LISP implementations might cap the effectiveness of manual 
	code optimizations.
    While Julia version of this package benefits from vectorization and permits 
    unsafe and fast memory manipulations, 
	REDUCE must operate on a higher and safer level of abstraction.
\end{description}

\begin{thebibliography}{BWK93}

\bibitem[BWK93]{tolstaya}
Thomas Becker, Volker Weispfenning, and Hainz Kredel.
\newblock {\em Gr{\"o}bner Bases}, volume 141 of {\em Graduate Texts in
  Mathematics}.
\newblock Springer, 1993.

\bibitem[G12]{GMP}
Torbj{\"o}rn Granlund and the GMP development team.
\newblock GNU MP: The GNU Multiple Precision Arithmetic Library.
\newblock \emph{http://gmplib.org/}, 2012

\bibitem[F99]{F4}
Jean~Charles Faug\'ere.
\newblock {A} new efficient algorithm for computing {G}r{\"o}bner bases ({F}4).
\newblock {\em Journal of Pure and Applied Algebra}, vol. 139(1), pages 61--88., 1999.

\end{thebibliography}

%\bibliography{f4}
%\bibliographystyle{alpha}

\end{document}
